\documentclass[12pt]{article}
\usepackage{geometry} 
\usepackage{indentfirst}
\usepackage{hyperref}
\usepackage{color}
\usepackage{comment}
\usepackage[pdftex]{graphicx}  
\usepackage{caption}
\usepackage{natbib}
\usepackage{mathtools}
\usepackage{units}
\usepackage{booktabs}
\usepackage{authblk}
\renewcommand{\baselinestretch}{1.5}
\geometry{a4paper} 
\bibliographystyle{apalike}

%----------------------------------------
%AUTHORS
%----------------------------------------
\title{... evidence for selection on maize centromere repeats}
\author[1]{Siddharth Bhadra-Lobo}
\author[1]{Christopher J. Fiscus}
\author[1]{Tyler V. Kent}
\author[1]{Nicole Soltis}
\author[1]{Timothy T. Yang} 
\author[1,2]{Jeffrey Ross-Ibarra\thanks{rossibarra@ucdavis.edu}}

\affil[1]{Department of Plant Sciences, University of California Davis}
\affil[2]{Center for Population Biology and Genome Center, University of California Davis}

\date{}

%-----------------------------------------------------------------------------------------------------------------
% BEGIN DOCUMENT
%-----------------------------------------------------------------------------------------------------------------
\begin{document}
\maketitle

\begin{abstract}
Abstract goes here.
\end{abstract}

\begin{footnotesize}
\noindent{\bf{\underline{GWAS}} results are available at \url{} }
\end{footnotesize}

\section*{Notes}

Will deleter this section later.  Some simple tips:
Bold can be done {\bf like this}.  Italics are \emph{easy} too.  Math formulas are also easy: $Area=\pi r^2$.
Here's how you enter a citation \cite{Wolfgruber2009}.
Here's another way to enter a citation \citep{Wolfgruber2009}.
All citations need to have a bibtex formatted entry in the file "references.bib" in this same repo. You can add comments to the file using "\%" which will not be shown in the final pdf.  

\section{Introduction}

\section{Materials and Methods}

\subsection{Estimation of CentC Abundance} %Nicole

\subsection{Genome-Wide Association Analyses} %Chris
Population structure was inferred using \emph{structure} software obtained from \url{http://pritchardlab.stanford.edu/structure.html}.  The basic algorithm of this software was first described by Pritchard, Stephens, \& Donnelly \cite{Pritchard2000} and extensions to the algorithm were described by Falush, Stephens \& Pritchard \cite{Falush2003}, \cite{Falush2007} and Hubisz, Falush, Stephens \& Pritchard \cite{Hubisz2009}. A randomly generated 10,000 SNP subset from Jiao et al dataset was used to generate q matrices and estimated natural log probability of the data for 1-20 maximum populations.  The length of burnin was 10000 and the number of MCMC reps after the burnin period was 20000.  The natural log probability of the data and delta K values for 3 trials were examined, revealing candidate K values at 3 and 12 assumed populations.    
% change wording in second sentence to be more readable.
% citation for Jiao dataset
% more info on burnin and MCMC reps
% method to determine K needs citation (Ln P(D) & delta K)
% general editing
% add section about GWAS using TASSEL 
\subsection{Imputation and Identity by Descent} %Tyler

\section{Results}

\section{Discussion}

\subsubsection*{Acknowledgments}
We would like to thank Anne Lorant for technical assistance, and Sofiane Mezmouk, blah, and blah for comments on earlier versions of this manuscript.  This work was funded in part by grant IOS-0922703 from the National Science Foundation.

%----------------------------------------
% REFERENCES
%----------------------------------------
\clearpage
\bibliography{references}

\end{document}